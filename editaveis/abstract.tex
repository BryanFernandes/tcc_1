\begin{resumo}[Abstract]
 \begin{otherlanguage*}{english}
The audiobook is an informational resource, be it in scientific, social, educational and political, that has promoted social inclusion of people in accessing information and knowledge. Through electronic means, they have also contributed to society by providing greater accessibility, agility in search of information, sharing, among other factors. The visually impaired has had an increasing participation with ace new information technologies, since people who have a visual impairment, access to information ends up being restricted. The audiobook as an informational resource created by modern technology can contribute to educational improvement and critical development by reading to non-seers. This feature can also be used by people who are blind, when it comes to cooperation with the training and regattas players as well as the use for instruction and learning a second language and literacy support. To better use and make it accessible to a larger number of people, a new format was specified oferendo support the navigation of audiobook content. The format makes use of data compression considering sharing and player that supports the new format has also been developed.

   \vspace{\onelineskip}
 
   \noindent 
   \textbf{Key-words}: \textit{Audiobooks}. Bookmarking Content. Ogg Vorbis. \textit{Open Source}.
 \end{otherlanguage*}
\end{resumo}
