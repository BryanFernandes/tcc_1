\chapter[Conclusão]{Conclusão}\label{cap5}

% Para evidenciar o cronograma aproximado, segue o gráfico de \textit{gantt} representado pela Figura \ref{cronograma} mostrando o desenvolvimento do projeto em um primeiro momento em que teve sua primeira semana iniciada no dia 16 de setembro de 2014. Nesta parte, o Editor foi desenvolvido e uma primeira versão do trabalho escrito finalizado. As entregas foram planejadas semanalmente. O gráfico mostra um aproximação pois algumas adaptações foram feitas no período de prototipagem.


% Em um segundo momento, a partir das atividades de 2015 dadas no início do 1º semestre da Universidade de Brasília \textit{campus} Gama, o player foi evoluído e totalmente adaptado ao formato comprimido especificado. Durante o semestre o Player passou por um longo período de adaptação, contemplando atividades de integração com o Editor, interpretação dos dados  decodificados e a correta visualização e execução de todas as funcionalidades possíveis. Paralelo a isso, o trabalho escrito era periodicamente atualizado.

\section{Trabalhos Futuros}

Como continuação deste trabalho, o próximo passo poderia ser a avaliação da solução tecnológica em um ambiente real. Uma metodologia de pesquisa de campo definida e utilizada como suporte a avaliação do uso da ferramenta por pessoas com deficiência visual. Os dados coletados e analisados podem gerar um relatório conclusivo sobre a proposta apontando seu desempenho como material didático de apoio aos deficientes visuais.