\chapter[Conclusão]{Conclusão}\label{cap5}

% Para evidenciar o cronograma aproximado, segue o gráfico de \textit{gantt} representado pela Figura \ref{cronograma} mostrando o desenvolvimento do projeto em um primeiro momento em que teve sua primeira semana iniciada no dia 16 de setembro de 2014. Nesta parte, o Editor foi desenvolvido e uma primeira versão do trabalho escrito finalizado. As entregas foram planejadas semanalmente. O gráfico mostra um aproximação pois algumas adaptações foram feitas no período de prototipagem.


% Em um segundo momento, a partir das atividades de 2015 dadas no início do 1º semestre da Universidade de Brasília \textit{campus} Gama, o player foi evoluído e totalmente adaptado ao formato comprimido especificado. Durante o semestre o Player passou por um longo período de adaptação, contemplando atividades de integração com o Editor, interpretação dos dados  decodificados e a correta visualização e execução de todas as funcionalidades possíveis. Paralelo a isso, o trabalho escrito era periodicamente atualizado.

Este trabalho demonstrou que a acessibilidade dificilmente é lembrada e faz parte do processo do desenvolvimento de uma solução tecnológica. Este fato é ainda mais evidente quando a solução não é estritamente focada para este público alvo. Tornar um sistema acessível para pessoas que possuam deficiência visual não é criar algo completamente novo mas é sabermos usar o que já existe e manejarmos de forma combinativa e integrada todo o aparato tecnológico disponível como suporte ao desenvolvimento de sistemas acessíveis. 

No início de um processo de desenvolvimento de software, requisitos de acessiblidade devem ser pensados e postos em prática. Por conta das restrições de acesso e uso de softwares existentes, muitas pessoas não conseguem fazer uso da solução tecnológica. O formato OGG, do qual foi derivado o formato especificado neste trabalho, é de código aberto e seus desenvolvedores não só permitem como incentivam a comunidade para contribuir com o formato. A partir desta plataforma, uma novo direcionamento é dado ao formato OGG. Como conclusão deste trabalho, novos pacotes foram criados e inseridos na estrutura do formato de áudio de forma a proporcionar a inserção de metadados e marcações do conteúdo do pacote de áudio. Com o uso deste novo modelo, além do áudio é possível adicionar informações referentes ao \textit{audiobook} (tais como: título, ano, editora, autor, entre outros) e navegar por capítulos e marcações. Deste modo, ao compartilhar o arquivo de \textit{audiobook}, o usuário não só terá o áudio mas a informação decorrente dele.

Com os requisitos necessários para tornar um Player acessível aos deficientes visuais, observou-se que o interfaceamento da aplicação foi completamente replanejado. Não somente a tela, mas o teclado recebeu novas funcionalidades também. Devido a tais modificações, constatou-se que os recursos de acessibilidade definem a interface e interação do Player com o usuário, de modo a garantir que o deficiente visual possa ter o conhecimento de todas as informações disponíveis. 

Espera-se que este trabalho contribua para uma nova forma de leitura de livros didáticos de maneira agradável e acessível para um número maior de pessoas dos que os livros impressos proporiconam. O arquivo gerado pelo Editor, faz uso da compressão de dados tornando seu tamanho reduzido o que facilita a disseminação dos \textit{audiobooks} pela Internet. O formato é \textit{open source} o que não trás um custo adicional para as editoras, diminuindo, consequentemente, o custo dos \textit{audiobooks}. Em um ambiente escolar, por exemplo, o ensino pode ser maximizado e pessoas com deficiência visual podem atingir seu potencial e fazerem sua plena contribuição a sociedade.

% O QtCreator é uma plataforma que proporciona uma vasto conjunto de ferramentas que ajudam no desenvolvimento de, dentre outras coisas, aplicações para desktops. O Player capaz de interpretar corretamente todos as informações contidas no formato especificado foi desenvolvido pelo uso dessa ferramenta. Os recursos de acessibilidade definem a interface e interação do Player com o usuário, de modo a garantir que o deficiente visual possa ter o conhecimento de todas as informações disponíveis. 


\section{Trabalhos Futuros}

Como continuação deste trabalho, o próximo passo seria a avaliação da solução tecnológica em um ambiente real. Uma metodologia de pesquisa de campo definida e utilizada como suporte a avaliação do uso da ferramenta por pessoas com deficiência visual. Os dados coletados e analisados podem gerar um relatório conclusivo sobre a proposta apontando seu desempenho como material didático de apoio aos deficientes visuais.