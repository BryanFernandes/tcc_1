\begin{apendicesenv}

\partapendices

\chapter{User Stories}\label{userstories}

Segue-se abaixo as User Stories descritas que define e prioriza os requisitos de acessibilidade para o Tocador. Aqui é possível identificar os atores, o cenário de possível interação com o utilizador do sistema, as funcionalidades e suas restrições, além dos pré-condições e pós-condições. Os critérios de aceitação para cada \textit{User Story} estão descritos no Apêndice B.

\begin{table}[ht]
\centering
\caption{US01 - Interação por Teclado}
\vspace{0.1cm}
%\rowcolor{1}{}{lightgray}
\begin{center}
\begin{tabularx}{\textwidth}{|lX|l|}

\hline 
\textbf{Narrativa da Story} & Interação por Teclado \\
\hline
& \\
\textit{Como} & deficiente visual \\
 & \\
\textit{Eu quero} & manter o controle do tocador por meio do teclado \\
& \\
\textit{Para que} & seja possível acessar e navegar pela informação contida no áudio. \\
& \\
\hline
\textit{Critérios de Aceitação} & ~\hyperref[tab:us01a1]{AC01}, ~\hyperref[tab:us01a2]{AC02}, ~\hyperref[tab:us01a3]{AC03}, ~\hyperref[tab:us01a4]{AC04}, ~\hyperref[tab:us01a5]{AC05}, ~\hyperref[tab:us01a6]{AC06}, ~\hyperref[tab:us01a7]{AC07}, ~\hyperref[tab:us01a8]{AC08}, ~\hyperref[tab:us01a9]{AC09}, ~\hyperref[tab:us01a10]{AC10}, ~\hyperref[tab:us01a11]{AC11}, ~\hyperref[tab:us01a12]{AC12}, ~\hyperref[tab:us01a23]{AC23} e ~\hyperref[tab:us01a24]{AC24} \\
\hline

\end{tabularx}
\end{center}
\label{tab:us01}
\end{table}

\begin{table}[ht]
\centering
\caption{US02 - Sintetizador para Leitura de Tela}
\vspace{0.1cm}
%\rowcolor{1}{}{lightgray}
\begin{center}
\begin{tabularx}{\textwidth}{|lX|l|}

\hline 
\textbf{Narrativa da Story} & Sintetizador para Leitura de Tela \\
\hline
 & \\
\textit{Como} & deficiente visual \\
 & \\
\textit{Eu quero} & ter uma resposta audível do que está na tela \\
 & \\
\textit{Para que} & saiba o que visualmente é mostrado da qual não consigo ver. \\
 & \\
\hline
\textit{Critérios de Aceitação} & Tabelas ~\hyperref[tab:us01a13]{AC13} e ~\hyperref[tab:us01a14]{AC14} \\
\hline

\end{tabularx}
\end{center}
\label{tab:us02}
\end{table}

\begin{table}[ht]
\centering
\caption{US03 - Contraste de tela}
\vspace{0.1cm}
%\rowcolor{1}{}{lightgray}
\begin{center}
\begin{tabularx}{\textwidth}{|lX|l|}
 
\hline 
\textbf{Narrativa da Story} & Contraste de tela \\
\hline
 & \\
\textit{Como} & deficiente visual com baixa visão \\
 & \\
\textit{Eu quero} & poder alterar o contraste da tela \\
 & \\
\textit{Para que} & eu consiga enxergar a informação visual e fazer distinção entre o conteúdo apresentado. \\
 & \\
\hline
\textit{Critérios de Aceitação} & ~\hyperref[tab:us01a15]{AC15} e ~\hyperref[tab:us01a25]{AC25} \\
\hline

\end{tabularx}
\end{center}
\label{tab:us03}
\end{table}


\begin{table}[ht]
\centering
\caption{US04 - Sintetizador para informação de conteúdo}
\vspace{0.1cm}
%\rowcolor{1}{}{lightgray}
\begin{center}
\begin{tabularx}{\textwidth}{|lX|l|}

\hline 
\textbf{Narrativa da Story} & Sintetizador para informação de conteúdo \\
\hline
 & \\
\textit{Como} & deficiente visual \\
 & \\
\textit{Eu quero} & ter uma resposta audível da informação do \textit{audiobook} \\
 & \\
\textit{Para que} & eu tenha conhecimento do autor, ano de publicação, editora, número de páginas, título, endereço e idioma do livro que está sendo falado. Também quero ter conhecido da marcação e nível da marcação atual. \\
 & \\
\hline
\textit{Critérios de Aceitação} & ~\hyperref[tab:us01a16]{AC16},  ~\hyperref[tab:us01a17]{AC17}, ~\hyperref[tab:us01a23]{AC23} e ~\hyperref[tab:us01a24]{AC24} \\
\hline

\end{tabularx}
\end{center}
\label{tab:us04}
\end{table}

\begin{table}[ht]
\centering
\caption{US05 - Sintetizador para ajuda}
\vspace{0.1cm}
%\rowcolor{1}{}{lightgray}
\begin{center}
\begin{tabularx}{\textwidth}{|lX|l|}

\hline 
\textbf{Narrativa da Story} & Sintetizador para ajuda \\
\hline
 & \\
\textit{Como} & deficiente visual \\
 & \\
\textit{Eu quero} & ter uma respota audível das funcionalidades da aplicação \\
 & \\
\textit{Para que} & eu seja ajudado quanto as funcionalidades e comandos disponíveis do Tocador. \\
 & \\
\hline
\textit{Critérios de Aceitação} & ~\hyperref[tab:us01a18]{AC18} e ~\hyperref[tab:us01a19]{AC19} \\
\hline

\end{tabularx}
\end{center}
\label{tab:us05}
\end{table}

\begin{table}[ht]
\centering
\caption{US06 - Destaque visual}
\vspace{0.1cm}
%\rowcolor{1}{}{lightgray}
\begin{center}
\begin{tabularx}{\textwidth}{|lX|l|}

\hline 
\textbf{Narrativa da Story} & Destaque visual \\
\hline
 & \\
\textit{Como} & deficiente visual com baixa visão \\
 & \\
\textit{Eu quero} & gostaria que os botões sob o mouse fossem destacados \\
 & \\
\textit{Para que} & eu saiba onde está o cursor e também identifique o botão antes de apertá-lo. \\
 & \\
\hline
\textit{Critérios de Aceitação} & ~\hyperref[tab:us01a20]{AC20} \\
\hline

\end{tabularx}
\end{center}
\label{tab:us06}
\end{table}

\begin{table}[ht]
\centering
\caption{US07 - Ampliador de tela}
\vspace{0.1cm}
%\rowcolor{1}{}{lightgray}
\begin{center}
\begin{tabularx}{\textwidth}{|lX|l|}

\hline 
\textbf{Narrativa da Story} & Ampliador de tela \\
\hline
 & \\
\textit{Como} & deficiente visual com baixa visão \\
 & \\
\textit{Eu quero} & que os textos sob o mouse sejam ampliados \\
 & \\
\textit{Para que} & seja tornar possível ou facilitar a leitura. \\
 & \\
\hline
\textit{Critérios de Aceitação} & ~\hyperref[tab:us01a21]{AC21} \\
\hline

\end{tabularx}
\end{center}
\label{tab:us07}
\end{table}

\begin{table}[ht]
\centering
\caption{US08 - Anúncio automático de botão sob o mouse}
\vspace{0.1cm}
%\rowcolor{1}{}{lightgray}
\begin{center}
\begin{tabularx}{\textwidth}{|lX|l|}

\hline 
\textbf{Narrativa da Story} & Anúncio automático de botão sob o mouse \\
\hline
 & \\
\textit{Como} & deficiente visual com baixa visão \\
 & \\
\textit{Eu quero} & que ao mover o mouse o botão sob ele seja anunciado \\
 & \\
\textit{Para que} & eu saiba qual o botão antes mesmo de clicá-lo. \\
 & \\
\hline
\textit{Critérios de Aceitação} & ~\hyperref[tab:us01a22]{AC22} \\
\hline

\end{tabularx}
\end{center}
\label{tab:us08}
\end{table}



\chapter{Critérios de Aceitação}\label{criterios}

%-------------------------------------------------------------------------------------
%		AC01: Executar audiobook
%-------------------------------------------------------------------------------------

\begin{table}[ht]
\centering
\caption{AC01 - Executar audiobook}
\vspace{0.1cm}
%\rowcolor{1}{}{lightgray}
\begin{center}
\begin{tabularx}{\textwidth}{|lX|l|}

\hline
\textbf{Critério para Aceitação} & Executar \textit{Audiobook} \\
\hline
 & \\			
\textit{Dado} & que o \textit{audiobook} não esteja sendo executado \\
 & \\
\textit{Quando} & a tecla \textit{space} for pressionada \\
 & \\
\textit{Então} & o \textit{audiobook} deverá ser executado. \\
 & \\
\hline
\textit{User Story} & ~\hyperref[tab:us01]{US01} \\
\hline

\end{tabularx}
\end{center}
\label{tab:us01a1}
\end{table}

%-------------------------------------------------------------------------------------
%		AC02: Pausar audiobook
%-------------------------------------------------------------------------------------

\begin{table}[ht]
\centering
\caption{AC02 - Pausar audiobook}
\vspace{0.1cm}
%\rowcolor{1}{}{lightgray}
\begin{center}
\begin{tabularx}{\textwidth}{|lX|l|}

\hline
\textbf{Critério para Aceitação} & Pausar Execução do \textit{Audiobook} \\
\hline
 & \\
\textit{Dado} & que o \textit{audiobook} esteja sendo executado \\
 & \\
\textit{Quando} & a tecla \textit{space} for pressionada \\
 & \\
\textit{Então} & a execução do \textit{audiobook} deverá ser pausada. \\
 & \\
\hline
\textit{User Story} & ~\hyperref[tab:us01]{US01} \\
\hline

\end{tabularx}
\end{center}
\label{tab:us01a2}
\end{table}

%-------------------------------------------------------------------------------------
%		AC03: Ir para próxima marcação de mesmo nível
%-------------------------------------------------------------------------------------

\begin{table}[ht]
\centering
\caption{AC03 - Ir para próxima marcação de mesmo nível}
\vspace{0.1cm}
%\rowcolor{1}{}{lightgray}
\begin{center}
\begin{tabularx}{\textwidth}{|lX|l|}

\hline
\textbf{Critério para Aceitação} & Ir para próxima marcação de mesmo nível \\
\hline
 & \\
\textit{Dado} & que o \textit{audiobook} esteja sendo executado ou não \\
 & \\
\textit{Quando} & a tecla \textit{direcional direita} for pressionada \\
 & \\
\textit{Então} & o Tocador deverá saltar para a próxima marcação de mesmo nível. Se for o final do \textit{audiobook} a execução deverá ser pausada. Caso o \textit{audiobook} esteja sendo executado ele deve continuar a execução a partir do salto. Caso o \textit{audiobook} não esteja sendo executado ele deve continuar pausado para onde houve o salto. \\
 & \\
\hline
\textit{User Story} & ~\hyperref[tab:us01]{US01} \\
\hline

\end{tabularx}
\end{center}
\label{tab:us01a3}
\end{table}

%-------------------------------------------------------------------------------------
%		AC04: Ir para marcação anterior de mesmo nível
%-------------------------------------------------------------------------------------

\begin{table}[ht]
\centering
\caption{AC04 - Ir para marcação anterior de mesmo nível}
\vspace{0.1cm}
%\rowcolor{1}{}{lightgray}
\begin{center}
\begin{tabularx}{\textwidth}{|lX|l|}

\hline
\textbf{Critério para Aceitação} & Ir para marcação anterior de mesmo nível \\
\hline
 & \\
\textit{Dado} & que o \textit{audiobook} esteja sendo executado ou não \\
 & \\
\textit{Quando} & a tecla \textit{direcional esquerda} for pressionada \\ 
 & \\
\textit{Então} & o Tocador deverá saltar para a marcação anterior de mesmo nível. Se for  o início do \textit{audiobook} a execução deverá continuar no estado atual a partir do tempo saltado. Caso o \textit{audiobook} esteja sendo executado ele deve continuar a execução a partir do salto. Caso o \textit{audiobook} não esteja sendo executado ele deve continuar pausado para onde houve o salto. \\
 & \\
\hline
\textit{User Story} & ~\hyperref[tab:us01]{US01} \\
\hline

\end{tabularx}
\end{center}
\label{tab:us01a4}
\end{table}

%-------------------------------------------------------------------------------------
%		AC05: Ir para próxima marcação
%-------------------------------------------------------------------------------------

\begin{table}[ht]
\centering
\caption{AC05 - Ir para próxima marcação}
\vspace{0.1cm}
%\rowcolor{1}{}{lightgray}
\begin{center}
\begin{tabularx}{\textwidth}{|lX|l|}

\hline
\textbf{Critério para Aceitação} & Ir para próxima marcação \\
\hline
 & \\
\textit{Dado} & que o \textit{audiobook} esteja sendo executado ou não \\
 & \\
\textit{Quando} & a tecla \textit{n} for pressionada \\
 & \\
\textit{Então} & o Tocador deverá saltar para a próxima marcação em relação ao tempo. Se for o final do \textit{audiobook} a execução deverá ser pausada. Caso o \textit{audiobook} esteja sendo executado ele deve continuar a execução a partir do salto. Caso o \textit{audiobook} não esteja sendo executado ele deve continuar pausado para onde houve o salto. \\
 & \\
\hline
\textit{User Story} & ~\hyperref[tab:us01]{US01} \\
\hline

\end{tabularx}
\end{center}
\label{tab:us01a5}
\end{table}

%-------------------------------------------------------------------------------------
%		AC06: Ir para marcação anterior
%-------------------------------------------------------------------------------------

\begin{table}[ht]
\centering
\caption{AC06 - Ir para marcação anterior}
\vspace{0.1cm}
%\rowcolor{1}{}{lightgray}
\begin{center}
\begin{tabularx}{\textwidth}{|lX|l|}

\hline
\textbf{Critério para Aceitação} & Ir para marcação anterior \\
\hline
 & \\
\textit{Dado} & que o \textit{audiobook} esteja sendo executado ou não \\
 & \\
\textit{Quando} & a tecla \textit{f} for pressionada \\
 & \\
\textit{Então} & o Tocador deverá saltar para a marcação anterior em relação ao tempo. Se for o início do \textit{audiobook} a execução deverá continuar no estado atual a partir do tempo saltado. Caso o \textit{audiobook} esteja sendo executado ele deve continuar a execução a partir do salto. Caso o \textit{audiobook} não esteja sendo executado ele deve continuar pausado para onde houve o salto. \\
 & \\
\hline
\textit{User Story} & ~\hyperref[tab:us01]{US01} \\
\hline

\end{tabularx}
\end{center}
\label{tab:us01a6}
\end{table}

%-------------------------------------------------------------------------------------
%		AC07: Subir nível
%-------------------------------------------------------------------------------------

\begin{table}[ht]
\centering
\caption{AC07 - Subir nível}
\vspace{0.1cm}
%\rowcolor{1}{}{lightgray}
\begin{center}
\begin{tabularx}{\textwidth}{|lX|l|}

\hline
\textbf{Critério para Aceitação} & Subir nível \\
\hline
 & \\
\textit{Dado} & que o \textit{audiobook} esteja sendo executado ou não \\
 & \\
\textit{Quando} & a tecla \textit{direcional para cima} for pressionada \\
 & \\
\textit{Então} & o Tocador deverá alterar a navegação para um nível acima. O estado atual de execução deverá ser mantido. \\
 & \\
\hline
\textit{User Story} & ~\hyperref[tab:us01]{US01} \\
\hline

\end{tabularx}
\end{center}
\label{tab:us01a7}
\end{table}

%-------------------------------------------------------------------------------------
%		AC08: Descer nível
%-------------------------------------------------------------------------------------

\begin{table}[ht]
\centering
\caption{AC08 - Descer nível}
\vspace{0.1cm}
%\rowcolor{1}{}{lightgray}
\begin{center}
\begin{tabularx}{\textwidth}{|lX|l|}

\hline
\textbf{Critério para Aceitação} & Descer nível \\
\hline
 & \\
\textit{Dado} & que o \textit{audiobook} esteja sendo executado ou não \\
 & \\
\textit{Quando} & a tecla \textit{direcional para baixo} for pressionada \\
 & \\
\textit{Então} & o Tocador deverá alterar a navegação para um nível abaixo. O estado atual de execução deverá ser mantido. \\
 & \\
\hline
\textit{User Story} & ~\hyperref[tab:us01]{US01} \\
\hline

\end{tabularx}
\end{center}
\label{tab:us01a8}
\end{table}

%-------------------------------------------------------------------------------------
%		AC09: Acessar metadados
%-------------------------------------------------------------------------------------

\begin{table}[ht]
\centering
\caption{AC09 - Acessar metadados}
\vspace{0.1cm}
%\rowcolor{1}{}{lightgray}
\begin{center}
\begin{tabularx}{\textwidth}{|lX|l|}

\hline
\textbf{Critério para Aceitação} & Acessar metadados \\
\hline
 & \\
\textit{Dado} & que o \textit{audiobook} esteja sendo executado ou não \\
 & \\
\textit{Quando} & a tecla \textit{i} for pressionada \\
 & \\
\textit{Então} & o metadado deverá ser informado ao usuário. O estado atual de execução deverá ser mantido. \\
 & \\
\hline
\textit{User Story} & ~\hyperref[tab:us01]{US01} \\
\hline

\end{tabularx}
\end{center}
\label{tab:us01a9}
\end{table}

%-------------------------------------------------------------------------------------
%		AC10: Solicitar metadados
%-------------------------------------------------------------------------------------

\begin{table}[ht]
\centering
\caption{AC10 - Solicitar metadados}
\vspace{0.1cm}
%\rowcolor{1}{}{lightgray}
\begin{center}
\begin{tabularx}{\textwidth}{|lX|l|}

\hline
\textbf{Critério para Aceitação} & Solicitar ajuda \\
\hline
 & \\
\textit{Dado} & que o \textit{audiobook} esteja sendo executado ou não \\
 & \\
\textit{Quando} & a tecla \textit{h} for pressionada \\
 & \\
\textit{Então} & o usuário deverá ser informado sobre as teclas funcionais do Tocador. O estado atual de execução deverá ser mantido. \\
 & \\
\hline
\textit{User Story} & ~\hyperref[tab:us01]{US01} \\
\hline

\end{tabularx}
\end{center}
\label{tab:us01a10}
\end{table}

%-------------------------------------------------------------------------------------
%		AC11: Alterar contraste
%-------------------------------------------------------------------------------------

\begin{table}[ht]
\centering
\caption{AC11 - Alterar contraste}
\vspace{0.1cm}
%\rowcolor{1}{}{lightgray}
\begin{center}
\begin{tabularx}{\textwidth}{|lX|l|}

\hline
\textbf{Critério para Aceitação} & Alterar contraste \\
\hline
 & \\
\textit{Dado} & que o \textit{audiobook} esteja sendo executado ou não \\
 & \\
\textit{Quando} & a tecla \textit{c} for pressionada \\
 & \\
\textit{Então} & o contraste de tela deverá ser alterado. O estado atual de execução deverá ser mantido. \\
 & \\
\hline
\textit{User Story} & ~\hyperref[tab:us01]{US01} \\
\hline

\end{tabularx}
\end{center}
\label{tab:us01a11}
\end{table}
	
%-------------------------------------------------------------------------------------
%		AC12: Tecla específica para cada botão
%-------------------------------------------------------------------------------------

\begin{table}[ht]
\centering
\caption{AC12 - Tecla específica para cada botão}
\vspace{0.1cm}
%\rowcolor{1}{}{lightgray}
\begin{center}
\begin{tabularx}{\textwidth}{|lX|l|}

\hline
\textbf{Critério para Aceitação} & Tecla específica para cada botão \\
\hline
 & \\
 & Cada um dos botões do Tocador deve estar associado a um tecla específica. \\
 & \\
\hline
\textit{User Story} & ~\hyperref[tab:us01]{US01} \\
\hline

\end{tabularx}
\end{center}
\label{tab:us01a12}
\end{table}

%-------------------------------------------------------------------------------------
%		AC13: Resposta audível
%-------------------------------------------------------------------------------------

\begin{table}[ht]
\centering
\caption{AC13 - Resposta audível}
\vspace{0.1cm}
%\rowcolor{1}{}{lightgray}
\begin{center}
\begin{tabularx}{\textwidth}{|lX|l|}

\hline
\textbf{Critério para Aceitação} & Resposta audível \\
\hline
 & \\
 & Cada elemento presente na tela deve ter uma resposta audível descrevendo o que ele é. \\
 & \\
\hline
\textit{User Story} & ~\hyperref[tab:us02]{US02} \\
\hline

\end{tabularx}
\end{center}
\label{tab:us01a13}
\end{table}

%-------------------------------------------------------------------------------------
%		AC14: Informar atualização de texto em tela
%-------------------------------------------------------------------------------------

\begin{table}[ht]
\centering
\caption{AC14 - Informar atualização de texto em tela}
\vspace{0.1cm}
%\rowcolor{1}{}{lightgray}
\begin{center}
\begin{tabularx}{\textwidth}{|lX|l|}

\hline
\textbf{Critério para Aceitação} & Informar atualização de texto em tela \\
\hline
 & \\
 & Para cada atualização de texto em tela o usuário deverá ser informado de forma audível. \\
 & \\
\hline
\textit{User Story} & ~\hyperref[tab:us02]{US02} \\
\hline

\end{tabularx}
\end{center}
\label{tab:us01a14}
\end{table}

%-------------------------------------------------------------------------------------
%		AC15: Alternativas mínimas de contraste 
%-------------------------------------------------------------------------------------

\begin{table}[ht]
\centering
\caption{AC15 - Alternativas mínimas de contraste}
\vspace{0.1cm}
%\rowcolor{1}{}{lightgray}
\begin{center}
\begin{tabularx}{\textwidth}{|lX|l|}

\hline
\textbf{Critério para Aceitação} & Alternativas mínimas de contraste \\
\hline
 & \\
 & O Tocador deve ter no mínimo uma opção alternativa de contraste. \\
 & \\
\hline
\textit{User Story} & ~\hyperref[tab:us03]{US03} \\
\hline

\end{tabularx}
\end{center}
\label{tab:us01a15}
\end{table}

%-------------------------------------------------------------------------------------
%		AC16: Informação falada
%-------------------------------------------------------------------------------------

\begin{table}[ht]
\centering
\caption{AC16 - Informação falada}
\vspace{0.1cm}
%\rowcolor{1}{}{lightgray}
\begin{center}
\begin{tabularx}{\textwidth}{|lX|l|}

\hline
\textbf{Critério para Aceitação} & Informação falada \\
\hline
 & \\
 & Para cada \textit{audiobook} deve ser possível escutar o título, o autor, o idioma, a editora, a localidade, o número de páginas e o ano de publicação. \\
 & \\
\hline
\textit{User Story} & ~\hyperref[tab:us04]{US04} \\
\hline

\end{tabularx}
\end{center}
\label{tab:us01a16}
\end{table}


%-------------------------------------------------------------------------------------
%		AC17: Disponibilidade da informação
%-------------------------------------------------------------------------------------

\begin{table}[ht]
\centering
\caption{AC17 - Disponibilidade da informação}
\vspace{0.1cm}
%\rowcolor{1}{}{lightgray}
\begin{center}
\begin{tabularx}{\textwidth}{|lX|l|}

\hline
\textbf{Critério para Aceitação} & Disponibilidade da informação \\
\hline
 & \\
 & O acesso a esta informação deve estar disponível a qualquer momento que o usuário solicitar. \\
 & \\
\hline
\textit{User Story} & ~\hyperref[tab:us04]{US04} \\
\hline

\end{tabularx}
\end{center}
\label{tab:us01a17}
\end{table}

%-------------------------------------------------------------------------------------
%		AC18: Anúncio dos comandos
%-------------------------------------------------------------------------------------

\begin{table}[ht]
\centering
\caption{AC18 - Anúncio dos comandos}
\vspace{0.1cm}
%\rowcolor{1}{}{lightgray}
\begin{center}
\begin{tabularx}{\textwidth}{|lX|l|}

\hline
\textbf{Critério para Aceitação} & Anúncio dos comandos \\
\hline
 & \\
 & Cada funcionalidade disponível deve ser pronunciada. \\
 & \\
\hline
\textit{User Story} & ~\hyperref[tab:us05]{US05} \\
\hline

\end{tabularx}
\end{center}
\label{tab:us01a18}
\end{table}

%-------------------------------------------------------------------------------------
%		AC19: Anúncio dos comandos
%-------------------------------------------------------------------------------------

\begin{table}[ht]
\centering
\caption{AC19 - Anúncio dos comandos}
\vspace{0.1cm}
%\rowcolor{1}{}{lightgray}
\begin{center}
\begin{tabularx}{\textwidth}{|lX|l|}

\hline
\textbf{Critério para Aceitação} & Anúncio dos comandos \\
\hline
 & \\
 & Cada funcionalidade disponível deve ser pronunciada. \\
 & \\
\hline
\textit{User Story} & ~\hyperref[tab:us05]{US05} \\
\hline

\end{tabularx}
\end{center}
\label{tab:us01a19}
\end{table}

%-------------------------------------------------------------------------------------
%		AC20: Cor específica para botões
%-------------------------------------------------------------------------------------

\begin{table}[ht]
\centering
\caption{AC20 - Cor específica para botões}
\vspace{0.1cm}
%\rowcolor{1}{}{lightgray}
\begin{center}
\begin{tabularx}{\textwidth}{|lX|l|}

\hline
\textbf{Critério para Aceitação} & Cor específica para botões \\
\hline
 & \\
 & Cada botão deve ter uma cor específica. \\
 & \\
\hline
\textit{User Story} & ~\hyperref[tab:us06]{US06} \\
\hline

\end{tabularx}
\end{center}
\label{tab:us01a20}
\end{table}

%-------------------------------------------------------------------------------------
%		AC21: Ampliando texto sob mouse
%-------------------------------------------------------------------------------------

\begin{table}[ht]
\centering
\caption{AC21 - Ampliar texto sob mouse}
\vspace{0.1cm}
%\rowcolor{1}{}{lightgray}
\begin{center}
\begin{tabularx}{\textwidth}{|lX|l|}

\hline
\textbf{Critério para Aceitação} & Ampliar texto sob mouse \\
\hline
 & \\
 & Os textos sob o mouse devem ser automaticamente ampliados. \\
 & \\
\hline
\textit{User Story} & ~\hyperref[tab:us07]{US07} \\
\hline

\end{tabularx}
\end{center}
\label{tab:us01a21}
\end{table}

%-------------------------------------------------------------------------------------
%		AC22: Anunciar botão sob mouse
%-------------------------------------------------------------------------------------

\begin{table}[ht]
\centering
\caption{AC22 - Anunciar texto sob mouse}
\vspace{0.1cm}
%\rowcolor{1}{}{lightgray}
\begin{center}
\begin{tabularx}{\textwidth}{|lX|l|}

\hline
\textbf{Critério para Aceitação} & Anunciar botão sob mouse \\
\hline
 & \\
 & O botão sob o mouse deve ser anunciado imediatamente. \\
 & \\
  & \\
\textit{Dado} & que o \textit{audiobook} esteja sendo executado ou não \\
 & \\
\textit{Quando} & a mouse passar por cima do botão \\
 & \\
\textit{Então} & a descrição da funcionalidade do botão deverá ser falada ao usuário. \\
 & \\
\hline
\textit{User Story} & ~\hyperref[tab:us08]{US08} \\
\hline

\end{tabularx}
\end{center}
\label{tab:us01a22}
\end{table}

%-------------------------------------------------------------------------------------
%		AC23: Acessar marcação
%-------------------------------------------------------------------------------------

\begin{table}[ht]
\centering
\caption{AC23 - Acessar marcação}
\vspace{0.1cm}
%\rowcolor{1}{}{lightgray}
\begin{center}
\begin{tabularx}{\textwidth}{|lX|l|}

\hline
\textbf{Critério para Aceitação} & Acessar marcação \\
\hline
 & \\			
\textit{Dado} & que o \textit{audiobook} esteja sendo executado ou não \\
 & \\
\textit{Quando} & a tecla \textit{m} for pressionada \\
 & \\
\textit{Então} & a marcação atual deverá ser informada ao usuário. O estado atual de execução deverá ser mantido. \\
 & \\
\hline
\textit{User Story} & ~\hyperref[tab:us01]{US01} e ~\hyperref[tab:us01]{US04} \\
\hline

\end{tabularx}
\end{center}
\label{tab:us01a23}
\end{table}

%-------------------------------------------------------------------------------------
%		AC24: Acessar nível da marcação
%-------------------------------------------------------------------------------------

\begin{table}[ht]
\centering
\caption{AC24 - Acessar nível da marcação}
\vspace{0.1cm}
%\rowcolor{1}{}{lightgray}
\begin{center}
\begin{tabularx}{\textwidth}{|lX|l|}

\hline
\textbf{Critério para Aceitação} & Acessar nível da marcação \\
\hline
 & \\			
\textit{Dado} & que o \textit{audiobook} esteja sendo executado ou não \\
 & \\
\textit{Quando} & a tecla \textit{n} for pressionada \\
 & \\
\textit{Então} & o nível da marcação atual deverá ser informada ao usuário. O estado atual de execução deverá ser mantido. \\
 & \\
\hline
\textit{User Story} & ~\hyperref[tab:us01]{US01} e ~\hyperref[tab:us01]{US04} \\
\hline

\end{tabularx}
\end{center}
\label{tab:us01a24}
\end{table}

%-------------------------------------------------------------------------------------
%		AC25: Alternar contraste de tela
%-------------------------------------------------------------------------------------

\begin{table}[ht]
\centering
\caption{AC25 - Alternar contraste de tela}
\vspace{0.1cm}
%\rowcolor{1}{}{lightgray}
\begin{center}
\begin{tabularx}{\textwidth}{|lX|l|}

\hline
\textbf{Critério para Aceitação} & Alternar contraste de tela \\
\hline
 & \\			
\textit{Dado} & que o \textit{audiobook} esteja sendo executado ou não \\
 & \\
\textit{Quando} & a tecla \textit{c} for pressionada \\
 & \\
\textit{Então} & o contraste da tela deverá ser alternado entre o constrate padrão e o alternativo. O estado atual de execução deverá ser mantido. \\
 & \\
\hline
\textit{User Story} & ~\hyperref[tab:us01]{US03} \\
\hline

\end{tabularx}
\end{center}
\label{tab:us01a25}
\end{table}

\end{apendicesenv}