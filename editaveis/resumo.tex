\begin{resumo}

%Neste trabalho é apresentada um proposta de evolução do trabalho proposto por \cite{herbert} e, para tal, será desenvolvido uma especificação de um formato livre e \textit{open source} com o objetivo de oferecer suporte de marcação de conteúdo para \textit{audiobooks}. Um Editor é desenvolvido no qual será responsável por codificar e decodicar os metadados de um arquivo do tipo Ogg Vorbis bem como inserir um novo pacote contendo as marcações de conteúdo. Tal proposta foi construída em cima das especificações do formato Ogg Vorbis e nas bibliotecas fornecidas pela fundação Xiph.org.

O \textit{audiobook} é um recurso informacional, seja ele no campo científico, social, educacional e político, que tem promovido a inclusão social de pessoas no acesso a informação e ao conhecimento. Através de meios eletrônicos, eles também tem contribuído com a sociedade proporcionando maior acessibilidade, agilidade na busca da informação, compartilhamento, dentre outros fatores. Os deficientes visuais tem tido uma participação cada vez maior junto ás novas tecnologias da informação, visto que, pessoas que possuem a limitação visual, o acesso a informação acaba sendo restrito. O \textit{audiobook}, como um recurso informacional gerado pelas novas tecnologias, pode contribuir para o aprimoramento educacional e desenvolvimento crítico através da leitura aos não-videntes. Este recurso pode também ser utilizado por pessoas que não possuem deficiência visual, no que tange a cooperação com a formação e regate de leitores como também na utilização para instrução e aprendizado de uma segunda língua e suporte a alfabetização. Para melhor utilização e tornar acessível para um número maior de pessoas, um novo formato foi especificado oferendo suporte a navegação do conteúdo do \textit{audiobook}. O formato faz uso da compressão de dados por considerar o compartilhamento e um player que dê suporte ao novo formato também foi desenvolvido.

 \vspace{\onelineskip}
    
 \noindent
 \textbf{Palavras-chaves}: Não-videntes. \textit{Audiobooks}. Marcação de conteúdo. Ogg Vorbis. \textit{Open Source}.
\end{resumo}
