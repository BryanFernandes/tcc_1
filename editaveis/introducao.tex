\chapter[Introdução]{Introdução}\label{cap1}

Os \textit{audiobooks} tem crescido no mercado e cada vez mais sendo usado pelas pessoas ao redor do mundo. Eles são amplamente difundidos pela internet e em sua maioria no formato MP3. Este formato tem sido difundido por armazenar áudio de alta qualidade e ocupar pouco espaço em memória devido ao uso de uma técnica de compressão de dados. Entretanto, com desvantagem, quaisquer marcações são ligadas diretamente ao tocador e não ao arquivo. Outro ponto é que não existem formatos \textit{open source} que ofereçam suporte para marcação de conteúdo e os que possuem são protegidos por leis de propriedade intelectual. Estas marcações são marcadores de posição que representam a estrutura lógica de um livro físico o que vem a ser capítulos, seções, parágrafos, versículos, entre outros. Estas são as informações levantadas por \cite{herbert} e a motivação de seu trabalho foi desenvolver uma especificação de um formato aberto para \textit{audiobooks} com suporte a marcadores de conteúdo pois não existe um formato de arquivo livre e de código aberto que tenha suporte a estes marcadores. Os objetivos foram alcançados mas melhorias foram sugeridas pois os arquivos gerados ocupavam muito espaço em memória.

Este trabalho visa reproduzir o trabalho feito por \cite{herbert} mas em um outro formato de arquivo que faz compressão de dados ao ponto de gerar arquivos menores viabilizando a proposta incialmente sugerida. Portanto, neste trabalho foi desenvolvida a especificação de um formato aberto para \textit{audiobooks} com suporte a marcadores de conteúdo e que faz uso da técnica de compressão de dados para o armazenamento dos dados.

O trabalho está organizado em capítulos. No próximo capítulo, é apresentado os objetivos pretendidos para o qual este trabalho foi motivado.

No capítulo \ref{cap3}, é apresentada toda a revisão bibliográfica onde foram revistos monografias, dissertações, livros, publicações em \textit{websites} e especificações necessários para o entendimento e desenvolvimento do projeto.

No capítulo \ref{cap4}, será restringida a proposta do trabalho.

No capítulo \ref{cap5}, serão apresentados todas as etapas realizadas para a especificação do formato e para o desenvolvimento do Editor bem como as ferramentas utilizadas com suporte no processo de desenvolvimento e pesquisa.

No capítulo \ref{cap6}, serão apresentados os resultados obtidos no projeto onde será mostrado o arquivo gerado pelo Editor.

E, por fim, no capítulo \ref{cap7}, será apresentado um cronograma onde estão definidas as atividades executadas durante o desenvolvimento deste trabalho.

%Este documento apresenta considerações gerais e preliminares relacionadas à redação de relatórios de Projeto de Graduação da Faculdade UnB Gama (FGA). São abordados os diferentes aspectos sobre a estrutura do trabalho, uso de programas de auxilio a edição, tiragem de cópias, encadernação, etc.
